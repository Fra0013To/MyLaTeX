%%% PACCHETTO PER SWITCH CASES IN newcommand %%%
\usepackage{xstring}
% Example:
% \newcommand{\tree}[2]{%
%     \IfEqCase{#1}{%
%         {a}{$\sqrt{#2}$}%
%         {b}{Hi}%
%         % you can add more cases here as desired
%     }[\PackageError{tree}{Undefined option to tree: #1}{}]%
% }%

%%% PACCHETTO PER IF/ELSE CASES IN newcommand
\usepackage{ifthen}
% In brief: \ifthenelse{test}{then clause}{else clause}



%%% FACCIAMO POSTO PER LE SCORCIATOIE SUCCESSIVE %%%

\newcommand{\dOld}{\d}
\newcommand{\sOld}{\s}
\newcommand{\lOld}{\l}
\newcommand{\rOld}{\r}
\newcommand{\vOld}{\v}
\newcommand{\tOld}{\t}
\newcommand{\bOld}{\b}
\newcommand{\cOld}{\c}
\newcommand{\kOld}{\k}
\newcommand{\COld}{\C}

%%% SEMPLIFICAZIONE DELLA SINTASSI E SCORCIATOIE %%%

\newcommand{\TENS}[1]{\mathop{\mathbf{#1}}\nolimits} % Tensori (lettere maiuscole)
\newcommand{\Dkl}[2]{D_{\mathrm{KL}}\left({#1}\parallel{#2}\right)}	% Divergenza di Kullback-Liebler
\newcommand{\Djs}[2]{D_{\mathrm{JS}}\left({#1}, {#2}\right)}	% Divergenza di Jensen-Shannon
\newcommand{\herm}[1]{{#1}^\dagger}	% Hermitiana di matrice
\newcommand{\trasp}[1]{{#1}^\top}	% Trasposta di matrice
\newcommand{\norm}[1]{\parallel{#1}\parallel}	% Norma
\newcommand{\eps}{\varepsilon}				%   eps: la vecchia varepsilon
\renewcommand{\d}{\mathrm{d}}					%   d:	differenziale
\newcommand{\f}{\frac}							%   f:	frazione
\newcommand{\tf}{\tfrac}						%  tf:	frazione piccola
\renewcommand{\s}{\sqrt}						%   s:	radice
\newcommand{\pd}[2]{\f{∂{#1}}{∂{#2}}}					%  pd:	derivata parziale
\newcommand{\pdd}[2]{\f{∂^2{#1}}{∂{#2}^2}}				% pdd:	derivata parziale seconda
\newcommand{\td}[2]{\f{\d{#1}}{\d{#2}}}					%  td:	derivata totale
\newcommand{\tdd}[2]{\f{\d^2{#1}}{\d{#2}^2}}				% tdd:	derivata totale seconda
\newcommand{\fd}[2]{\f{δ{#1}}{δ{#2}}}					%  fd:	derivata funzionale
\newcommand{\pq}{\overleftrightarrow{∂}}				%  pq:  derivata parziale $bidirezionale$
\renewcommand{\l}{\left}						%   l:	left
\renewcommand{\r}{\right}						%   r:	right
\renewcommand{\v}{\ensuremath{\boldsymbol}}				%   v:	vettore (grassetto) [MEGLIO \boldsymbol O \pmb ?]
\renewcommand{\t}{\text}						%   t:	text
\newcommand{\I}{\indices}						%   I:	indici tensoriali (pacchetto tensor)
\newcommand{\h}{\ensuremath{\hbar}}					%   h:	h tagliato
\newcommand{\ccall}{\ensuremath{\mathcal}}				%   c:  calligrafico
\renewcommand{\C}{\ensuremath{\mathbb{C}}}				%   C:  insieme dei numeri complessi
% \newcommand{\C}{\ensuremath{\mathbb{C}}}				%   C:  insieme dei numeri complessi

\newcommand{\N}{\ensuremath{\mathbb{N}}}				%   N:  insieme dei numeri naturali
\newcommand{\R}{\ensuremath{\mathbb{R}}}				%   R:  insieme dei numeri Reali
\newcommand{\Q}{\ensuremath{\mathbb{Q}}}				%   Q:  insieme dei numeri Razionali
\newcommand{\Z}{\ensuremath{\mathbb{Z}}}				%   Z:  insieme dei numeri Interi
\renewcommand{\k}[1]{\ensuremath{\mathinner{|{#1}\rangle}}}		%   k:	|⋅>
\renewcommand{\b}[1]{\ensuremath{\mathinner{\langle{#1}|}}}		%   b:	<⋅|
\newcommand{\bk}[1]{\ensuremath{\mathinner{\langle{#1}\rangle}}}	%  bk:  <⋅>
\newcommand{\bkt}[2]{\mathinner{\langle{#1}|{#2}\rangle}}		% bkt:  <⋅|⋅>
\newcommand{\B}[1]{\left\langle#1\right|}				%   B:  |⋅>     scaling
\newcommand{\K}[1]{\left|#1\right\rangle}				%   K:  <⋅|     scaling
\newcommand{\BK}[1]{\left\langle{#1}\right\rangle}			%  BK:  <⋅>     scaling
\newcommand{\BKT}[2]{\left\langle{#1}\mid{#2}\right\rangle}		% BKT:  <⋅|⋅>   scaling
\newcommand{\wt}{\widetilde}						%  td:	widetilde
\newcommand{\ol}{\overline}
\newcommand{\bsy}{\boldsymbol}		                % abbreviation for math boldsymbol 

%%% OPERATORI MATEMATICI %%% (aggiungere ad ogni necessità)
\DeclareMathOperator{\Tr}{Tr}
\DeclareMathOperator{\tr}{tr}
\DeclareMathOperator{\Var}{Var}
\DeclareMathOperator{\Cov}{Cov}
\DeclareMathOperator{\Corr}{Corr}
\DeclareMathOperator{\argmin}{arg\,min}
\DeclareMathOperator{\argmax}{arg\,max}
\DeclareMathOperator{\E}{\ensuremath{\mathbb{E}}}				%   E:  Valore Atteso
\DeclareMathOperator{\sigm}{sigm}
\DeclareMathOperator{\relu}{relu}
\DeclareMathOperator{\elu}{elu}
\DeclareMathOperator{\softp}{softp}
\DeclareMathOperator{\swish}{swish}
\DeclareMathOperator{\softm}{softmax}
\DeclareMathOperator{\depth}{depth}

%%%%%%%%%%%%% T O D O %%%%%%%%%%%%%%%
% includere \bb{} al posto di \mathbb{}. Contestualmente scegliere uno da http://www.latex-community.org/forum/viewtopic.php?f=5&t=2238 (link in fondo). Deve funzionare anche \bb{1}

%%% Se si vuole nabla vettore... %%%
%\let\nablino\nabla
%\renewcommand{\nabla}{\boldsymbol{\nablino}}
