%%% BIBLATEX %%%
\usepackage[backend=biber,style=numeric-comp,citestyle=numeric,sorting=nty,natbib,maxbibnames=10]{biblatex}
%%%%%%%%%%%%%%%%%%%%%%%


%%% GRAFICHE %%%
\usepackage{graphicx}
% TIKZ %
\usepackage{tikz}
% PGFPLOTS %
\usepackage{pgfplots}
\pgfplotsset{compat=1.6}%{width=10cm, compat=1.9} % retrocompatibilità e proprietà varie plot
\usepgfplotslibrary{groupplots}
% VELOCIZZAZIONE/RETROCOMPATIBILITA' PGFPLOTS (CREA PDF CHE POI CARICA) - NON SEMPRE FUNZIONA %
% \usepgfplotslibrary{external}
% \tikzexternalize
% GRAFI CON TIKZ (richiede tikz)%
\usepackage{pgf}
\usetikzlibrary{positioning,shapes,arrows,fit,calc,automata}
% PACKAGE FOR DRAWING NNs (UP-TO-DATE PACKAGE NOT ON CTAN BUT ON https://github.com/battlesnake/neural) %
\usepackage{neuralnetwork}
% PACKAGE FOR NETWORK DRAWINGS (ALSO 3D!!)
\usepackage{tikz-network}

%%% LINGUA %%%
\usepackage[english]{babel}

%%% A4WIDE OPTION %%%%%%%
\usepackage{a4wide}

%%% SPAZIO INTORNO A SEZIONE & SIMILI %%%
%\usepackage{titlesec}
%%FUNZIONAMENTO GENERALE:
%%\titlespacing*{<command>}{<left>}{<before-sep>}{<after-sep>}
%% ESEMPI (USATO IN TESI PhD):
%\titlespacing*{\section}
%{0pt}{3\baselineskip}{1.5\baselineskip}

%%% SUPPORTO UNICODE %%%
%\usepackage{ucs}      % SEMBRA NON SERVIRE
% \usepackage[utf8x]{inputenc}
% IN TOPTESI NON FUNZIONANO... (A COSA SERVIVANO?)
% \SetUnicodeOption{mathletters}
% \SetUnicodeOption{autogenerated}
%%%%%%%%%%%%%%%%%%%%%%%

%%% SIMBOLI %%%
\usepackage{amsmath, amsthm, amsfonts, amssymb}
\usepackage{mathabx}   % upup-..-dowdow arrows
\usepackage[makeroom]{cancel}    % cancellation symbol
\usepackage{tipa}      % fonetica
\usepackage{textcomp}  % caratteri extra
\usepackage{latexsym}  % simboli vari
\usepackage{mathrsfs}  % caratteri corsivi (agg. io); per utilizzarlo: \mathscr{•}
%%%%%%%%%%%%%


%%% LUNGHEZZA DELL'INDENTAZIONE (se 0 non c'è)%%%
\setlength{\parindent}{3mm}


%% NOMENCLATURA %%%
%\usepackage[intoc]{nomencl}
%\makenomenclature
% Importazione opzioni per gruppi di nomenclature
% \input{ReferencesAndNomenclatures/GroupsNomenclature.tex}
% PER STAMPARE LA NOMENCLATURA:
%- fare un file .tex con all'interno varie voci del tipo: \nomenclature[group]{symbol}{description}
%- importare il file .tex (dove si vuole)
%- scrivere il comando \printnomenclature dove si vuole che venga stampata la nomenclatura
% ATTENZIONE: in toptesi la nomenclatura viene "letta" solo se inserita nel main matter!
%%%%%%%%%%%%%%%%%%%


%%% ENUMITEM CON NUMERAZIONE "STILE SEZIONI" %%%
% per numerazioni 1.1,1.2,... nell'ambiente ENUMERATE quando ne metto uno dentro all'altro
\renewcommand{\labelenumii}{\theenumii}
\renewcommand{\theenumii}{\theenumi.\arabic{enumii}.}

%credo oppure (metterlo nella cartella come per 'mcode'):
\usepackage{enumitem}% http://ctan.org/pkg/enumitem
% e utilizzare così:
%\begin{enumerate}
%  \item First
%  \begin{enumerate}[label*=\arabic*.]
%    \item Second
%    \item Third
%  \end{enumerate}
%  \item Fourth
%\end{enumerate}

%%%%%%%%%%%%%%%%%%%%%%%%%%%%%%%%%%%%%%%%%%%


%%% PACCHETTO PER LINK A REF  %%%
\usepackage{hyperref}
\hypersetup{
	%bookmarks=true,         % show bookmarks bar?
	%unicode=false,          % non-Latin characters in Acrobat’s bookmarks
	%pdftoolbar=true,        % show Acrobat’s toolbar?
	%pdfmenubar=true,        % show Acrobat’s menu?
	%pdffitwindow=false,     % window fit to page when opened
	%pdfstartview={FitH},    % fits the width of the page to the window
	%pdftitle={My title},    % title
	%pdfauthor={Author},     % author
	%pdfsubject={Subject},   % subject of the document
	%pdfcreator={Creator},   % creator of the document
	%pdfproducer={Producer}, % producer of the document
	%pdfkeywords={keyword1, key2, key3}, % list of keywords
	%pdfnewwindow=true,      % links in new PDF window
	colorlinks=true,       % false: boxed links; true: colored links
	linkcolor=blue,%red,          % color of internal links (change box color with linkbordercolor)
	citecolor=red,%green,        % color of links to bibliography
	filecolor=magenta,      % color of file links
	urlcolor=cyan           % color of external links
}
%%%%%%%%%%%%%%%%%%%%%%%%%%%%%%

%%% CLEVEREF PACKAGE FOR \Cref, \cref, ecc. %%%
\usepackage{cleveref}

%%% PERSONALIZZAZIONE STILE TABELLE %%%
% PACCHETTO PER HLINE IN TABELLE
\usepackage{hhline}
% FATT. MOLTIPLICATIVO ALTEZZA TABELLE (Valore standard 1.2)
\renewcommand\arraystretch{1.5}
% ALTRI PACCHETTI PER TABELLE (MULTIRIGHE E COLORI)
\usepackage{multirow}
\usepackage{colortbl} 
%%%%%%%%%%%%%%%%%%%%%%%%%%%%%%%%%%



%%% PACCHETTI PER ALGORITMI IN PSEUDOCODICE %%%
\usepackage{algpseudocode}
\usepackage{algorithm}
%%%%%%%%%%%%%%%%%%%%%%%%%%%%%%%%%%%%%%%%%%


%%% PACCHETTO PER CORREZIONI %%%
\usepackage{cancel}
%%%%%%%%%%%%%%%%%%%%%%%%%%%%%


%%% ---- COLORAZIONE DIMOSTRAZIONI (commentare per non colorarle) ---
%\usepackage{etoolbox}
%\usepackage{xcolor}
%\usepackage{amsthm}
%\AtBeginEnvironment{proof}{\color{lightgray}}
%%% -----------------------------------------



