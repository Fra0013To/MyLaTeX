%Stile CORSIVO per i 'newtheorem' successivi

%ambiente 'teo' con la scritta 'Theorem' e numerazione subordinata alla sezione
\newtheorem{teo}{Theorem}[section]

%ambiente 'prop' con la scritta 'Proposition' e numerazione subordinata alla sezione
\newtheorem{prop}{Proposition}[section]

%ambiente 'cor' con la scritta 'Corollary' e numerazione uguale al teorema
\newtheorem{cor}[teo]{Corollary}

%ambiente 'lem' con la scritta 'Lemma' e numerazione subordinata alla sezione
\newtheorem{lem}{Lemma}[section]

%ambiente 'defin' con la scritta 'Definition' e numerazione subordinata alla sezione
\newtheorem{defin}{Definition}[section]

%ambiente 'sch' con la scritta 'Scheme' e numerazione subordinata alla sezione
\newtheorem{sch}{Scheme}[section]

%ambiente 'prin' con la scritta 'Principle' e numerazione subordinata alla sezione
\newtheorem{prin}{Principle}[section]

%ambiente 'ex' con la scritta 'Exercise' e numerazione subordinata alla sezione
\newtheorem{ex}{Exercise}[section]

%ambiente 'notaz' con la scritta 'Notation' e numerazione subordinata alla sezione
\newtheorem{notaz}{Notation}[section]

%ambiente 'alg' con la scritta 'Algoritmo' e numerazione subordinata alla sezione
%\newtheorem{alg}{Algorithm}[section]



%Stile diverso (NON CORSIVO) per i 'newtheorem' successivi
\theoremstyle{definition}

%ambiente 'oss' con la scritta 'Observation' e numerazione subordinata alla sezione
\newtheorem{oss}{Observation}[section]

%ambiente 'oss' con la scritta 'Observation' e numerazione subordinata alla sezione
\newtheorem{rem}{Remark}[section]

%ambiente 'prog' con la scritta 'Program' e numerazione subordinata alla sezione
\newtheorem{prog}{Program}[section]

%ambiente 'es' con la scritta 'Example' e numerazione subordinata alla sezione
\newtheorem{es}{Example}[section]

%%%%%%%%%%%%%%%%%%%%%%%%%%%%%%%%%%%%%%%%%%%%%%%%%%%%%%%%%%%
